\section{Data Access}

\paragraph{Ground Truth}
To preserve user privacy,
the bottom line is that the raw data must not be collected.

The initial phase of FedCampus operates on participants' health data recorded by
the Huawei Watch Fit 2.
100 smartwatches of this model are purchased to be lent to the participants.

In order to conduct research on the data,
our algorithm needs to access the data from the smartwatch.

In the natural life cycle of the collected health data,
the data are first synchronized from the participants' smartwatches to
the Huawei HealthKit on the their smartphone through Bluetooth.
This is important in two ways:
first, we do not need to develop any software to run on the smartwatches,
only the smartphones; second,
we require the participants to regularly conduct this synchronization process
themselves, which counts into management complexity.

We deploy different strategies to access the data from the Huawei HealthKit on
Android and iOS.
On Android, Huawei provides the Huawei HealthKit API to access the data.
On iOS, Apple restricts third-party access to private data,
and unifies such access to Apple HealthKit,
so we have to first synchronize the data from the Huawei HealthKit to
the Apple HealthKit, then access the data via Apple HealthKit.

\section{The \fedcampus Application}

The major challenge of the \fedcampus Application is to
support both Android and iOS devices.
This is necessary because our participants at Duke Kunshan University are
a mix of Android and iOS users,
and we want to allow as many of them to participate as possible.
However, Android and iOS applications are incompatible,
and developing separate applications for these two platforms would be
a laborious task because
it would mean we have to maintain two separate codebases and
duplicate any changes.

To support mobile development on both Android and iOS,
we leverage the Flutter application framework.
Instead of developing two separate applications for Android and iOS,
by using Flutter,
only a single codebase is needed to for both the Android and iOS versions of
the application.
Therefore, we implement as much of the \fedcampus Application on Flutter as
possible to maximize our code reuse and minimize our development effort.
