%! TEX program = xelatex
\documentclass[11pt,a4paper,oneside]{report}

\usepackage{amsmath,amssymb}
\usepackage{parskip}
\usepackage{graphicx}
\usepackage{xcolor}
\usepackage[a4paper,margin=1in]{geometry}
\usepackage{longtable,booktabs,array}

\usepackage{titlesec}
\titleformat{\chapter}[display]
{\sffamily\bfseries\huge}
{\Large \chaptertitlename\ \thechapter}
{2ex}
{\titlerule
    \vspace{1ex}%
    \filright\MakeUppercase}
[\vspace{1ex}%
    \titlerule]
\titleformat{\section}{\normalfont\sffamily\Large\bfseries}{\thesection}{1em}{}
\titleformat{\subsection}{\normalfont\sffamily\large\bfseries}{\thesubsection}{1em}{}
\titleformat{\subsubsection}{\normalfont\sffamily\normalsize\bfseries}{\thesubsubsection}{1em}{}

\usepackage[style=numeric-comp, url=false]{biblatex}
\addbibresource{bibfile.bib}

\usepackage{fontspec}
\setmainfont{TeX Gyre Termes}
\setsansfont{TeX Gyre Termes}

\usepackage{xeCJK}
\setCJKmainfont{Songti SC}
\setCJKsansfont{Songti SC}
\setCJKmonofont{Songti SC}

\newcommand{\instructions}[1]{{\color{orange}\itshape #1}}
\renewcommand{\instructions}[1]{} % Uncomment this to hide the instructions.

\usepackage{upquote}
\usepackage[allcolors=blue,colorlinks=true]{hyperref}
\usepackage{xurl}
\usepackage{microtype}
\usepackage{bookmark}
\usepackage{calc}
\usepackage{etoolbox}

\urlstyle{same}

\usepackage{pifont}  % for the checkmark/crossmark
\usepackage{graphicx}

\usepackage{algorithmic}
\usepackage[ruled]{algorithm2e}

\usepackage{xspace}
\newcommand{\fedcampus}{\textsc{FedCampus}\xspace}
\newcommand{\fedkit}{\textsc{FedKit}\xspace}

\begin{document}


% Author name (capitalized in its regular way)
\newcommand{\authorname}{Sichang He}

% Title (cannot exceed three lines)
\newcommand{\thetitle}{Integrating Federated Learning for \fedcampus,
    A Privacy-Preserving Data Platform for Smart Campus
}

% Date of submission with normal capitalization. Use the format January 29, 2022.
\newcommand{\submissiondate}{March 7, 2024}

% Mentor: First Name Last Name (normal capitalization)
\newcommand{\mentor}{Bing Luo}

% Academic Unit (no abbreviations)
\newcommand{\academicunit}{Division of Natural and Applied Sciences}

%%%%%%%%%%%%%%%%%%%%%%%%%%%%%%%%%%%%%%%%%%%%%%%%%%%%%%%%%%%%%%%%%%%%%%%%%%%%%%%%

%% DO NOT CHANGE DIRECTLY THE CONTENTS OF THE TITLE PAGE.
%% TO CUSTOMIZE THE TITLE PAGE CHANGE THE DEFINITIONS OF THE COMMANDS
%% \authorname, \thetitle, \submissiondate, \mentor, \academicunit

\begin{titlepage}

    \vspace*{\bigskipamount}

    \begin{center}
        {\sffamily\LARGE\bfseries\MakeUppercase\thetitle\par}

        \bigskip

        by

        \bigskip

        {\Large \authorname}

        \bigskip

        Signature Work Product, in partial fulfillment of the \\
        Duke Kunshan University Undergraduate Degree Program

        \bigskip

        \emph{\submissiondate}

        \bigskip

        Signature Work Program \\
        Duke Kunshan University

    \end{center}

    \vfill

    \textbf{\textsf{APPROVALS}}

    \bigskip\bigskip\bigskip
    \hrule

    Mentor: \mentor, \academicunit

    \bigskip\bigskip\bigskip
    \hrule

    Marcia B. France, Dean of Undergraduate Studies

\end{titlepage}

%%%%%%%%%%%%%%%%%%%%%%%%%%%%%%%%%%%%%%%%%%%%%%%%%%%%%%%%%%%%%%%%%%%%%%%%%%%%%%%%

% Front matter
\clearpage
\pagenumbering{roman}

%%%%%%%%%%%%%%%%%%%%%%%%%%%%%%%%%%%%%%%%%%%%%%%%%%%%%%%%%%%%%%%%%%%%%%%%%%%%%%%%

\setcounter{tocdepth}{0} % Only top-level units (chapters) should appear in the TOC
\tableofcontents

%%%%%%%%%%%%%%%%%%%%%%%%%%%%%%%%%%%%%%%%%%%%%%%%%%%%%%%%%%%%%%%%%%%%%%%%%%%%%%%%

\chapter*{Abstract}
\addcontentsline{toc}{chapter}{Abstract}

% Abstract in English

\instructions{Abstract (English): 150 -- 200 words. An abstract is a brief
    statement of the problem or the purpose of the research. It should indicate
    the theoretical work or experimental plan used, summarize principal findings
    of the research, and point out major conclusions. Appropriate safety
    information should be included when applicable. This should be the section
    you write last to be sure that it accurately reflects the content of the
    document.}

\fedcampus endeavors to establish a privacy-preserving data platform for
university campuses,
extracting valuable insights while preserving participants' privacy.
This initial phase focuses on accessing personal health data on participants'
smartphones and executing cross-platform federated learning. Federated learning,
a privacy-preserving machine learning technique,
enables model training across multiple edge devices without centralizing raw
data. However,
challenges brought by \fedcampus' advanced requirements for real-world federated
learning on smartphones, including cross-platform support,
dependency on application updates, and interdisciplinary collaboration,
were not addressed by existing federated learning systems. \fedkit,
our innovative software development kits,
address these challenges through a machine learning model pipeline and machine
learning operations support.
Experiments were conducted in both lab-based settings and on The \fedcampus
Application, a smartphone application deployed among students, staff,
and faculty at Duke Kunshan University. Results underscored the effectiveness of
\fedkit and the feasibility of \fedcampus' overall approach,
paving the way for future advancements towards a comprehensive data platform.

\vspace{4\bigskipamount}

% Abstract in Chinese

\instructions{摘要(中文):150 - 200
    字。摘要是对问题或研究目的的简要说明。说明所使用的理论工作或实验计划,总结研究的主要发现,
    并指出主要结论。适用时应包括适当的安全信息。这应该是您最后编写的部分,
    以确保它准确反映文档的内容。}


%%%%%%%%%%%%%%%%%%%%%%%%%%%%%%%%%%%%%%%%%%%%%%%%%%%%%%%%%%%%%%%%%%%%%%%%%%%%%%%%

\chapter*{Acknowledgements}
\label{acknowledgements}
\addcontentsline{toc}{chapter}{Acknowledgements}

\instructions{Individuals and organizations who helped with the research project
    and provided financing are thanked in a paragraph of the thesis. Do not
    include individual titles in the acknowledgments. However, it is
    appropriate to state grant numbers and sponsors. Examples would like
    SELF, SRS, SW Grants, etc.}

This project is supported by Duke Kunshan University Undergraduate Office via
the Summer Research Scholars Program. Thanks to all \fedcampus team members and
alumni; your hard work made this project possible.
Special thanks to our chief developer Beilong Tang and our shepherd Jiaqi Shao.

\newpage

%%%%%%%%%%%%%%%%%%%%%%%%%%%%%%%%%%%%%%%%%%%%%%%%%%%%%%%%%%%%%%%%%%%%%%%%%%%%%%%%

% Add captions to your figures for them to appear in the List of Figures.
% Alternatively, comment out the next two lines if there are no tables 
% in your document.
\addcontentsline{toc}{chapter}{List of Figures}
\setcounter{tocdepth}{1}
\listoffigures\newpage

%%%%%%%%%%%%%%%%%%%%%%%%%%%%%%%%%%%%%%%%%%%%%%%%%%%%%%%%%%%%%%%%%%%%%%%%%%%%%%%%

% Add captions to your tables for them to appear in the List of Tables.
% Alternatively, comment out the next two lines if there are no tables 
% in your document.
\addcontentsline{toc}{chapter}{List of Tables}
\setcounter{tocdepth}{1}
\listoftables\newpage

%%%%%%%%%%%%%%%%%%%%%%%%%%%%%%%%%%%%%%%%%%%%%%%%%%%%%%%%%%%%%%%%%%%%%%%%%%%%%%%%

% Main matter
\clearpage
\pagenumbering{arabic}

%%%%%%%%%%%%%%%%%%%%%%%%%%%%%%%%%%%%%%%%%%%%%%%%%%%%%%%%%%%%%%%%%%%%%%%%%%%%%%%%

\chapter{Introduction}
\label{introduction}

\instructions{This section includes a clear statement of the problem and the
    reasons for studying it.~Provide a detailed yet concise background
    discussion of the problem and the significance, scope, and limits of the
    work. Outline what has been done previously by citing truly pertinent
    literature but do not include a general survey of semi-relevant
    literature.~ State how your work differs from earlier work in the field
    and demonstrate the continuity from the previous work to your own.}

\section{Federated Learning}

Federated learning is a promising technique to collaboratively train
shared machine learning models on edge devices,
thereby preserving user privacy and eliminating the need to
transfer raw data~\cite{mcmahan2017communication}.

However, most existing federated learning systems~\cite[e.g.,][]{
    bonawitz2019towards,ma2019paddlepaddle,liu2021fate,openfl_citation,
} primarily cater to running simulations on desktop computers or cloud servers,
which is distinct from our real-world use case.
These systems cannot be directly leveraged to support our experiments.

\section{Machine Learning Operations}

% TODO: Improve this and cite.
Machine learning operations (MLOps) is a set of practices similar to
development operations (DevOps).
In development operations,
the software development process and the deployment process are integrated,
so that new software changes made by the development team can be
automatically deployed into production.
Similarly, in machine learning operations,
changes in the machine learning algorithms can be automatically deployed into
production.
Machine learning operations is desirable when the development velocity on
the machine learning algorithms is high,
and feedbacks from the production environment are important to improve
the machine learning algorithms.

\section{Federated Learning on Smartphones}

Several systems have emerged to fulfill the demand for federated learning on
mobile devices, but still exhibit important limitations,
as shown in Table~\ref{tbl:fn-systems}.

Specifically, existing systems fall short in their support for
cross-platform federated learning on Android and iOS devices.
FedML~\cite{he2020fedml},
Microsoft's Project Florida~\cite{madrigal2023project},
and the FedScale benchmark~\cite{lai2022fedscale} primarily support Android.
On the other hand,
Flower~\cite{beutel2020flower,mathur2021ondevice} and
Syft~\cite{ryffel2018generic,Ziller2021,hall2021syft}
provide software development kits for both Android and iOS.
However, Flower software development kits only provide communication support,
and do not provide support for on-device training;
it also provides incompatible communication formats in
the Android software development kit versus the iOS software development kit,
so that models from the two platforms cannot be aggregated together.
Syft only uses CPU for on-device training, and lacks hardware acceleration;
this is undesirable because on-device training is slow and energy-consuming
without hardware acceleration.
Therefore, building a federated learning system for smart campus applications is
still an open challenge.

\begin{table}\begin{center}\label{tbl:fn-systems}
    \begin{tabular}{lccccc}
Functionality         & FedML     & Project Florida   & Flower    & Syft      & \textbf{\fedkit} \\
\hline
Android-Only          & \ding{51} & \ding{51} & \ding{51} & \ding{51} & \ding{51}       \\
iOS-Only              & \ding{55} & \ding{55} & \ding{51} & \ding{51} & \ding{51}       \\
Cross-Platform Aggregation  & \ding{55} & \ding{55} & \ding{55} & \ding{51} & \ding{51}       \\
\hline
Training Acceleration & \ding{51} & \ding{51} & \ding{51} & \ding{55} & \ding{51}       \\
MLOps                 & \ding{51} & \ding{51} & \ding{55} & \ding{55} & \ding{51}       \\
Open-Source Backend   & \ding{55} & \ding{55} & \ding{51} & \ding{51} & \ding{51}       \\
    \end{tabular}
    \caption{Functionality Comparison Among Federated Learning Systems With
        Smartphone Support.
    }
\end{center}\end{table}

In addition to seamless cross-platform support,
customizable and continuous deployment in production is crucial.
While FedML and Project Florida support machine learning operations,
their proprietary backends limit full customization.
In contrast, open-source solutions like Flower and Syft enable
full customization
but are more suitable for single experiments rather than continuous deployment.

To support \fedcampus' use case, we developed our own custom solution \fedkit,
a software developed kit designed to enable \textbf{cross-platform}
federated learning research on Android and iOS devices.
\fedkit introduces two major contributions:

\begin{enumerate}
\item \fedkit provides \textbf{modularized libraries} to convert Python-based models,
    and train and aggregate them across platforms.
    These libraries can serve as a foundational resource for
    on-device FL or even ML systems.
\item The FL workflow in \fedkit
    enables flexible \textbf{machine learning operations} from
    the backend in production.
    This innovation encourages other open-source solutions to
    rival proprietary services in deployment support.
\end{enumerate}

\section{Federated Analytics}

\section{Healthcare Data Analytics}


%%%%%%%%%%%%%%%%%%%%%%%%%%%%%%%%%%%%%%%%%%%%%%%%%%%%%%%%%%%%%%%%%%%%%%%%%%%%%%%%

\chapter{Methods}
\label{methods}

\instructions{This section is obviously discipline specific so use the
    nomenclature that is common for your discipline. However, this section
    should provide sufficient detail about the materials and the methods
    used so that other experienced workers can repeat the experiment and
    obtain comparable results. Cite the appropriate literature when using a
    standard method or protocol and give only the details needed. Identify
    the materials used in the research. For example, computer systems used,
    mathematical theorems exploited, etc.; give information on the purity of
    all chemicals and reagents employed in the research; include the
    chemical/biological names of all compounds and chemical formulas of
    substances that are new or uncommon. Use standard systematic
    nomenclature to unambiguously define well-established compounds,
    processes, equipment, etc.}

\section{General Federated Learning Procedure}

\begin{algorithm}
    \caption{General Federated Learning Procedure}
    \label{algo:general-procedure}
\ForEach{Training iteration}{
Distribute the global model to all clients\;
\ForEach{Client}{
    Train the local model using local training data\;
    Send the trained local model back to the central server\;
}
The central server aggregates all local models into a new global model\;
}
\end{algorithm}

The general procedure of federated learning on mobile devices involves a
distributed system with two parties: the clients and the central server.
The process unfolds through training iterations, with four phases per iteration,
as illustrated in Algorithm~\ref{algo:general-procedure}. In each iteration,
the central server distributes the global model to clients,
who then train the global model locally using their individual data.
Trained local models are sent back to the central server,
and these models are aggregated to generate an updated global model.
This process repeats for successive iterations.
In the context of federated learning on mobile devices,
clients are typically mobile applications on Android or iOS devices,
and the central server is a remote server on the Internet,
as depicted in Fig.~\ref{fig:general-fl}.

\begin{figure}\begin{center}
\includegraphics[width=0.7\textwidth]{general_fl.png}
    \caption{General Federated Learning Procedure on Smartphones.}
    \label{fig:general-fl}
\end{center}\end{figure}

The specific implementations of federated learning on mobile devices may vary,
but they share a common foundation. The FedAvg algorithm,
presented with federated learning itself~\cite{mcmahan2017communication},
stands as the most typical federated learning algorithm. In FedAvg,
a fixed number of $C$ out of $K$ clients are selected for local training in each
iteration.
The local training aims to minimize the loss $L$ of the model for each client's
local data partition $P_k (k \in \{1, 2, \dots, K\})$:
\begin{equation}
\min_{w_k} L(P_k;w_k),
\end{equation}
starting from the parameters $w^{(t)}$ of the latest global model from the
server, and scheduled for a fixed number of $E$ epochs.
To optimize the global model for the entire training dataset $\bigcup_k P_k$,
FedAvg aims to minimize the weighted average of local losses
\begin{equation}
\min_{w} \frac{\sum_k |P_k|L(P_k;w)}{\sum_k |P_k|},
\text{ where }|P_k|\text{ is the size of }P_k,
\end{equation}
by computing the weighted average of local models' weights
\begin{equation}
w^{(t+1)}=\sum_k \frac{|P_k|}{\sum_k |P_k|}w_k^{(t+1)}
\end{equation}
to update the global model.
This iteration is repeated until model convergence or experiment termination.
FedAvg has proven effective and practical in various experimental scenarios,
benchmarked against earlier algorithms in data center
settings~\cite{bonawitz2019towards}.

\section{Machine Learning Operations}

% TODO: Improve this and cite.
Machine learning operations (MLOps) is a set of practices similar to
development operations (DevOps).
In development operations,
the software development process and the deployment process are integrated,
so that new software changes made by the development team can be
automatically deployed into production.
Similarly, in machine learning operations,
changes in the machine learning algorithms can be automatically deployed into
production.
Machine learning operations is desirable when the development velocity on
the machine learning algorithms is high,
and feedbacks from the production environment are important to improve
the machine learning algorithms.


%%%%%%%%%%%%%%%%%%%%%%%%%%%%%%%%%%%%%%%%%%%%%%%%%%%%%%%%%%%%%%%%%%%%%%%%%%%%%%%%

\chapter{Design}
\label{design}

\section{Data Access}

The initial phase of FedCampus operates on participants' health data recorded by
the Huawei Watch Fit 2.
100 smartwatches of this model are purchased to be lent to the participants.

As a ground truth, to preserve user privacy,
the bottom line is that the raw data must not be collected.
However, in order to conduct research on the data,
our algorithms need to access the data from the smartwatch.

To deploy algorithms without collecting raw data,
we need to deploy the algorithms on users' edge devices and
access the data there.
In the natural life cycle of the collected health data,
the data are first synchronized from the participants' smartwatches to
the Huawei Health Kit~\cite{huaweihealthkit} on
their smartphone through Bluetooth.
This is important in two ways:
first, we do not need to develop any software to run on the smartwatches,
only the smartphones; second,
we require the participants to regularly conduct this synchronization process
themselves, which counts into management complexity.

We deploy different strategies to access the data from the Huawei Health Kit on
Android and iOS.
On Android, Huawei provides the Huawei Health Kit API to access the data.
On iOS, Apple restricts third-party access to private data,
and unifies such access to Apple HealthKit~\cite{applehealthkit},
so we have to first synchronize the data from the Huawei Health Kit to
the Apple HealthKit, then access the data via Apple HealthKit.

\section{The \fedcampus Application}

The major challenge of the \fedcampus Application is to
support both Android and iOS devices.
This is necessary because our participants at Duke Kunshan University are
a mix of Android and iOS users,
and we want to allow as many of them to participate as possible.
However, Android and iOS applications are incompatible,
and developing separate applications for these two platforms would be
a laborious task because
it would mean we have to maintain two separate codebases and
duplicate any changes.

To support mobile development on both Android and iOS,
we leverage the Flutter application framework.
Instead of developing two separate applications for Android and iOS,
by using Flutter,
only a single codebase is needed to for both the Android and iOS versions of
the application.
Therefore, we implement as much of the \fedcampus Application on Flutter as
possible to maximize our code reuse and minimize our development effort.

\section{Designing \fedkit}

\subsection{On-Device Training}

The FedScale benchmark~\cite{lai2022fedscale} takes an interesting approach to
run TensorFlow on Android---it uses the Termux Application to
create a UNIX shell environment.

\subsection{Machine Learning Operations}



%%%%%%%%%%%%%%%%%%%%%%%%%%%%%%%%%%%%%%%%%%%%%%%%%%%%%%%%%%%%%%%%%%%%%%%%%%%%%%%%

\chapter{Implementation}
\label{implementation}

\section{Accessing the Data}

On Android, we use the Huawei HealthKit API Java package to request
Huawei HealthKit for the data.
This API allows access to all the collected raw health data points.
We call the API from using Kotlin,
a programming language compatible with Java,
and aggregate the data into a daily summary.
% TODO: What are the data types.

\section{\fedkit}

\subsection{Cross-Platform Federated Learning Model Pipeline}

\subsubsection{Model Conversion}

\subsubsection{On-Device Training}

\subsubsection{Cross-Platform Model Aggregation}

\subsection{Machine Learning Operations}

\subsubsection{Continuous Cross-Platform Model Delivery}


%%%%%%%%%%%%%%%%%%%%%%%%%%%%%%%%%%%%%%%%%%%%%%%%%%%%%%%%%%%%%%%%%%%%%%%%%%%%%%%%

\chapter{Results}
\label{results}

\instructions{Summarize the data collected in this section, and their
    statistical treatment. Include only relevant data, but give sufficient
    detail to justify the conclusions. It is appropriate in this section to
    use equations, figures, and tables to display your data. Extensive, but
    relevant data, should be reserved for an appendix where it is identified
    as supporting information.}

\instructions{The table or figure must follow as closely as possible after the
    paragraph in which it is referenced. Titles/captions should be kept
    brief.}

\instructions{
    \section{Examples}

    Here is some inline math, $x^2 > 1$, and some display math
    \begin{equation}
        \int_0^1 x^2 \, dx
    \end{equation}
    And this is how to cite an article \cite{Zhang2021} or a book \cite{Axler2020}.

    \begin{table}[htbp]
        \centering
        \begin{tabular}{@{}llll@{}}
            \toprule
            \emph{Replace} & \emph{With} & \emph{Your} & \emph{Table} \\
            \midrule
                           &             &             &              \\
                           &             &             &              \\
            \bottomrule
        \end{tabular}
        \caption{Parameters for the optimization of the principal component analysis for
            olive oil adulteration.}
        \label{tbl:2}
    \end{table}


    \begin{figure}[htbp]
        \centering
        \includegraphics[height=4cm]{btc.jpg}
        \caption{The notorious BTC (Brandon The Cat).}
        \label{fig:1}
    \end{figure}
}

\section{Participant Requirements}

Our involved data access setup results in involved participant requirements.

In practice, accessing the data via the Huawei HealthKit on Android only works
well for participants using Huawei smartphones.
On Huawei smartphones,
the operating system automatically periodically synchronizes health data from
the connected Huawei smartwatch to the Huawei HealthKit.
Therefore, when these participants launch our FedCampus Application,
the app can immediately access the latest data from the Huawei HealthKit.

However, on non-Huawei Android smartphones,
the synchronization from the smartwatches only happens when
the Huawei Health app is on the foreground,
unless participants install Huawei Management System Core (HMS Core),
a background service app.
Alpha testers reported that the HMS Core is battery-consuming,
and is sometimes shut down by the operating system on smartphones made by
Xiaomi and other companies.
Additionally, Huawei apps are not available on the Google Play Store,
so some participants also have to install the Huawei AppGallery Application.
This makes an undesirable experience where the participants are required to
install four different apps.
Despite the inconveniences, once the participants have installed the apps,
they do not need to conduct this process again.

On iOS, participants can download the Huawei Health app from the App Store,
but it does not synchronize data from the Huawei smartwatch in the background.
Participants have to manually open the Huawei Health app,
and leave it in the foreground to synchronize the data both from
the smartwatches to the Huawei HealthKit,
and from the Huawei HealthKit to the Apple HealthKit.
It takes a significant amount of time for Huawei HealthKit to
synchronize the data to Apple HealthKit,
especially for the sleep time data;
as a result, it is common for the FedCampus Application to miss sleep data as
it could not read it from Apple HealthKit.

Due to these practical limitations,
we first prioritized Huawei smartphone users, then iPhone users,
for our initial launch.


%%%%%%%%%%%%%%%%%%%%%%%%%%%%%%%%%%%%%%%%%%%%%%%%%%%%%%%%%%%%%%%%%%%%%%%%%%%%%%%%

\chapter{Discussion}
\label{discussion}

\instructions{The discussion section is where you interpret and compare the
    results. The objective is to point out the features and limitations of
    the work. Relate your results to current knowledge in the field and to
    the original purpose for undertaking the project.}

Our data access setup is involved and it is desirable to be simplified.
However, we do not have much control over this process because of
our overall approach.
The data are collected by the proprietary Huawei smartwatches,
and synchronized to the proprietary Huawei HealthKit.
This means that we do not have control over when and how the data is
collected or synchronized.
In the future, it would be desirable to have a more open data access setup,
where we control the data collection and synchronization process.


%%%%%%%%%%%%%%%%%%%%%%%%%%%%%%%%%%%%%%%%%%%%%%%%%%%%%%%%%%%%%%%%%%%%%%%%%%%%%%%%

\chapter{Conclusions}
\label{conclusions}

\instructions{This section is written to put the interpretation of the results
    into the context of the original problem.~ Do not repeat the discussion
    points or include irrelevant material. The conclusion should be based on
    the evidence presented.}

This work marks the initial step towards realizing \fedcampus,
a privacy-preserving data platform for smart campuses.
Solutions were developed to access health data from smartwatches,
facilitate federated learning across smartphones of different operating systems,
and seamlessly integrate functionalities into a smartphone application for
participants. The introduction of \fedkit,
featuring an innovative machine learning model pipeline and elegant machine
learning operations support,
addresses pivotal challenges encountered during cross-platform federated
learning.

Our experiments,
conducted on The \fedcampus Application with participation from tens of
students, staff, and faculty,
substantiated the efficacy of \fedkit and affirmed the feasibility of our
holistic approach.
While encountering with significant challenges in data access and iOS on-device
training,
the system's inherent flexibility enables it for future enhancements.
As we continue to refine and extend \fedcampus,
addressing these challenges will contribute to the evolution of a robust and
comprehensive privacy-preserving data platform for smart campuses.


%%%%%%%%%%%%%%%%%%%%%%%%%%%%%%%%%%%%%%%%%%%%%%%%%%%%%%%%%%%%%%%%%%%%%%%%%%%%%%%%

\chapter*{References}
\label{references}
\addcontentsline{toc}{chapter}{References}


\instructions{Many bibliographic styles are acceptable for publications
    in the natural sciences. This template uses a numeric style defined in biblatex
    and that is common in Physics, Mathematics, and Computer Science papers.}

\printbibliography[heading=none]

%%%%%%%%%%%%%%%%%%%%%%%%%%%%%%%%%%%%%%%%%%%%%%%%%%%%%%%%%%%%%%%%%%%%%%%%%%%%%%%%

\appendix

\chapter{Federated Analytics}
\label{app:fa}

\instructions{This template can be viewed on Overleaf at
    \url{https://www.overleaf.com/read/hxjcgtkhjqcd}.  If you have an Overleaf
    account (either free or paid) you can copy this template to start a new
    Overleaf project.  If you do not want an Overleaf account you can install
    TeX on your computer and download the template files from Overleaf.  }

Federated analytics is a decentralized approach that aggregates statistical
information from multiple sources without compromising individual user data.
To preserve user privacy,
the \fedcampus Application employs a privacy-preserving mechanism known as
differential privacy. Before uploading raw data to our servers,
Gaussian white noise is added to individual user data.
This ensures that the aggregated statistics remain robust while preventing the
identification of specific users.

The decision to utilize differential privacy aligns with our commitment to
protecting user information. Even with this privacy-preserving measure,
the group statistics derived from federated analytics retain their significance.
They provide valuable insights into overall trends and patterns within the
\fedcampus user community.

\end{document}
