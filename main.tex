%! TEX program = lualatex
\documentclass[11pt,a4paper,oneside]{report}

\usepackage{amsmath,amssymb}
\usepackage{parskip}
\usepackage{graphicx}
\usepackage{xcolor}
\usepackage[a4paper,margin=1in]{geometry}
\usepackage{longtable,booktabs,array}

\usepackage{titlesec}
\titleformat{\chapter}[display]
  {\sffamily\bfseries\huge}
  {\Large \chaptertitlename\ \thechapter}
  {2ex}
  {\titlerule
   \vspace{1ex}%
   \filright\MakeUppercase}
  [\vspace{1ex}%
\titlerule]
\titleformat{\section}{\normalfont\sffamily\Large\bfseries}{\thesection}{1em}{}
\titleformat{\subsection}{\normalfont\sffamily\large\bfseries}{\thesubsection}{1em}{}
\titleformat{\subsubsection}{\normalfont\sffamily\normalsize\bfseries}{\thesubsubsection}{1em}{}

\usepackage[style=numeric-comp, url=false]{biblatex}
\addbibresource{bibfile.bib}

\usepackage{fontspec}
\setmainfont{Times New Roman}

\newcommand{\instructions}[1]{{\color{orange}\itshape #1}}

\usepackage{upquote}
\usepackage[allcolors=blue,colorlinks=true]{hyperref}
\usepackage{xurl}
\usepackage{microtype}
\usepackage{bookmark}
\usepackage{calc}
\usepackage{etoolbox}

\urlstyle{same}

\begin{document}


% Author name (capitalized in its regular way)
\newcommand{\authorname}{Sichang He}

% Title (cannot exceed three lines)
\newcommand{\thetitle}{FedCampus: A Privacy-Preserving Data Platform for Smart Campus}

% Date of submission with normal capitalization. Use the format January 29, 2022.
\newcommand{\submissiondate}{March 7, 2024}

% Mentor: First Name Last Name (normal capitalization)
\newcommand{\mentor}{Bing Luo}

% Academic Unit (no abbreviations)
\newcommand{\academicunit}{Division of Natural and Applied Sciences}

%%%%%%%%%%%%%%%%%%%%%%%%%%%%%%%%%%%%%%%%%%%%%%%%%%%%%%%%%%%%%%%%%%%%%%%%%%%%%%%%

%% DO NOT CHANGE DIRECTLY THE CONTENTS OF THE TITLE PAGE.
%% TO CUSTOMIZE THE TITLE PAGE CHANGE THE DEFINITIONS OF THE COMMANDS
%% \authorname, \thetitle, \submissiondate, \mentor, \academicunit

\begin{titlepage}

\vspace*{\bigskipamount}

\begin{center}
{\sffamily\LARGE\bfseries\MakeUppercase\thetitle\par}

\bigskip

by

\bigskip

{\Large \authorname}

\bigskip

Signature Work Product, in partial fulfillment of the \\
Duke Kunshan University Undergraduate Degree Program

\bigskip

\emph{\submissiondate}

\bigskip

Signature Work Program \\
Duke Kunshan University

\end{center}

\vfill

\textbf{\textsf{APPROVALS}}

\bigskip\bigskip\bigskip
\hrule

Mentor: \mentor, \academicunit

\bigskip\bigskip\bigskip
\hrule

Marcia B. France, Dean of Undergraduate Studies

\end{titlepage}

%%%%%%%%%%%%%%%%%%%%%%%%%%%%%%%%%%%%%%%%%%%%%%%%%%%%%%%%%%%%%%%%%%%%%%%%%%%%%%%%

% Front matter
\clearpage
\pagenumbering{roman}

%%%%%%%%%%%%%%%%%%%%%%%%%%%%%%%%%%%%%%%%%%%%%%%%%%%%%%%%%%%%%%%%%%%%%%%%%%%%%%%%

\setcounter{tocdepth}{0} % Only top-level units (chapters) should appear in the TOC
\tableofcontents

%%%%%%%%%%%%%%%%%%%%%%%%%%%%%%%%%%%%%%%%%%%%%%%%%%%%%%%%%%%%%%%%%%%%%%%%%%%%%%%%

\chapter*{Abstract}
\addcontentsline{toc}{chapter}{Abstract}

% Abstract in English

\instructions{Abstract (English): 150 -- 200 words. An abstract is a brief
statement of the problem or the purpose of the research. It should indicate
the theoretical work or experimental plan used, summarize principal findings
of the research, and point out major conclusions. Appropriate safety
information should be included when applicable. This should be the section
you write last to be sure that it accurately reflects the content of the 
document.}

\vspace{4\bigskipamount}

% Abstract in Chinese

\instructions{摘要(中文):150 - 200
字。摘要是对问题或研究目的的简要说明。说明所使用的理论工作或实验计划,总结研究的主要发现,
并指出主要结论。适用时应包括适当的安全信息。这应该是您最后编写的部分,
以确保它准确反映文档的内容。}


%%%%%%%%%%%%%%%%%%%%%%%%%%%%%%%%%%%%%%%%%%%%%%%%%%%%%%%%%%%%%%%%%%%%%%%%%%%%%%%%

\chapter*{Acknowledgements}
\label{acknowledgements}
\addcontentsline{toc}{chapter}{Acknowledgements}

\instructions{Individuals and organizations who helped with the research project
and provided financing are thanked in a paragraph of the thesis. Do not
include individual titles in the acknowledgments. However, it is
appropriate to state grant numbers and sponsors. Examples would like
SELF, SRS, SW Grants, etc.}

\newpage

%%%%%%%%%%%%%%%%%%%%%%%%%%%%%%%%%%%%%%%%%%%%%%%%%%%%%%%%%%%%%%%%%%%%%%%%%%%%%%%%

% Add captions to your figures for them to appear in the List of Figures.
% Alternatively, comment out the next two lines if there are no tables 
% in your document.
\addcontentsline{toc}{chapter}{List of Figures}
\setcounter{tocdepth}{1}
\listoffigures\newpage

%%%%%%%%%%%%%%%%%%%%%%%%%%%%%%%%%%%%%%%%%%%%%%%%%%%%%%%%%%%%%%%%%%%%%%%%%%%%%%%%

% Add captions to your tables for them to appear in the List of Tables.
% Alternatively, comment out the next two lines if there are no tables 
% in your document.
\addcontentsline{toc}{chapter}{List of Tables}
\setcounter{tocdepth}{1}
\listoftables\newpage

%%%%%%%%%%%%%%%%%%%%%%%%%%%%%%%%%%%%%%%%%%%%%%%%%%%%%%%%%%%%%%%%%%%%%%%%%%%%%%%%

% Main matter
\clearpage
\pagenumbering{arabic}

%%%%%%%%%%%%%%%%%%%%%%%%%%%%%%%%%%%%%%%%%%%%%%%%%%%%%%%%%%%%%%%%%%%%%%%%%%%%%%%%

\chapter{Introduction}
\label{introduction}

\instructions{This section includes a clear statement of the problem and the
reasons for studying it.~Provide a detailed yet concise background
discussion of the problem and the significance, scope, and limits of the
work. Outline what has been done previously by citing truly pertinent
literature but do not include a general survey of semi-relevant
literature.~ State how your work differs from earlier work in the field
and demonstrate the continuity from the previous work to your own.}

%%%%%%%%%%%%%%%%%%%%%%%%%%%%%%%%%%%%%%%%%%%%%%%%%%%%%%%%%%%%%%%%%%%%%%%%%%%%%%%%

\chapter{Material and Methods}
\label{material-and-methods}

\instructions{This section is obviously discipline specific so use the
nomenclature that is common for your discipline. However, this section
should provide sufficient detail about the materials and the methods
used so that other experienced workers can repeat the experiment and
obtain comparable results. Cite the appropriate literature when using a
standard method or protocol and give only the details needed. Identify
the materials used in the research. For example, computer systems used,
mathematical theorems exploited, etc.; give information on the purity of
all chemicals and reagents employed in the research; include the
chemical/biological names of all compounds and chemical formulas of
substances that are new or uncommon. Use standard systematic
nomenclature to unambiguously define well-established compounds,
processes, equipment, etc.}

%%%%%%%%%%%%%%%%%%%%%%%%%%%%%%%%%%%%%%%%%%%%%%%%%%%%%%%%%%%%%%%%%%%%%%%%%%%%%%%%

\chapter{Results}
\label{results}

\instructions{Summarize the data collected in this section, and their
statistical treatment. Include only relevant data, but give sufficient
detail to justify the conclusions. It is appropriate in this section to
use equations, figures, and tables to display your data. Extensive, but
relevant data, should be reserved for an appendix where it is identified
as supporting information.}

\instructions{The table or figure must follow as closely as possible after the
paragraph in which it is referenced. Titles/captions should be kept
brief.}

\section{Examples}

Here is some inline math, $x^2 > 1$, and some display math
\begin{equation}
  \int_0^1 x^2 \, dx
\end{equation}
And this is how to cite an article \cite{Zhang2021} or a book \cite{Axler2020}.

\begin{table}[htbp]
\centering
\begin{tabular}{@{}llll@{}}
\toprule
\emph{Replace} & \emph{With} & \emph{Your} & \emph{Table} \\
\midrule
& & & \\
& & & \\
\bottomrule
\end{tabular}
\caption{Parameters for the optimization of the principal component analysis for
olive oil adulteration.}
\label{tbl:2}  
\end{table}


\begin{figure}[htbp]
\centering
\includegraphics[height=4cm]{btc.jpg}
\caption{The notorious BTC (Brandon The Cat).}
\label{fig:1}
\end{figure}

%%%%%%%%%%%%%%%%%%%%%%%%%%%%%%%%%%%%%%%%%%%%%%%%%%%%%%%%%%%%%%%%%%%%%%%%%%%%%%%%

\chapter{Discussion}
\label{discussion}

\instructions{The discussion section is where you interpret and compare the
results. The objective is to point out the features and limitations of
the work. Relate your results to current knowledge in the field and to
the original purpose for undertaking the project.}

%%%%%%%%%%%%%%%%%%%%%%%%%%%%%%%%%%%%%%%%%%%%%%%%%%%%%%%%%%%%%%%%%%%%%%%%%%%%%%%%

\chapter{Conclusions}
\label{conclusions}

\instructions{This section is written to put the interpretation of the results
into the context of the original problem.~ Do not repeat the discussion
points or include irrelevant material. The conclusion should be based on
the evidence presented.}

%%%%%%%%%%%%%%%%%%%%%%%%%%%%%%%%%%%%%%%%%%%%%%%%%%%%%%%%%%%%%%%%%%%%%%%%%%%%%%%%

\chapter*{References}
\label{references}
\addcontentsline{toc}{chapter}{References}


\instructions{Many bibliographic styles are acceptable for publications 
in the natural sciences. This template uses a numeric style defined in biblatex
and that is common in Physics, Mathematics, and Computer Science papers.}

\printbibliography[heading=none]

%%%%%%%%%%%%%%%%%%%%%%%%%%%%%%%%%%%%%%%%%%%%%%%%%%%%%%%%%%%%%%%%%%%%%%%%%%%%%%%%

\appendix

\chapter{Additional Material}
\label{appendix-a}

This template can be viewed on Overleaf at \url{https://www.overleaf.com/read/hxjcgtkhjqcd}.
If you have an Overleaf account (either free or paid) you can copy this template to start a new Overleaf project.
If you do not want an Overleaf account you can install TeX on your computer and download the template files from Overleaf.

\end{document}
