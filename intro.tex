\section{Federated Learning}

Federated learning is a promising technique to collaboratively train
shared machine learning models on edge devices,
thereby preserving user privacy and eliminating the need to
transfer raw data~\cite{mcmahan2017communication}.

However, most existing federated learning systems~\cite[e.g.,][]{
    bonawitz2019towards,ma2019paddlepaddle,liu2021fate,openfl_citation,
} primarily cater to running simulations on desktop computers or cloud servers,
which is distinct from our real-world use case.
These systems cannot be directly leveraged to support our experiments.

\section{Machine Learning Operations}

% TODO: Improve this and cite.
Machine learning operations (MLOps) is a set of practices similar to
development operations (DevOps).
In development operations,
the software development process and the deployment process are integrated,
so that new software changes made by the development team can be
automatically deployed into production.
Similarly, in machine learning operations,
changes in the machine learning algorithms can be automatically deployed into
production.
Machine learning operations is desirable when the development velocity on
the machine learning algorithms is high,
and feedbacks from the production environment are important to improve
the machine learning algorithms.

\section{Federated Learning on Smartphones}

Several systems have emerged to fulfill the demand for federated learning on
mobile devices, but still exhibit important limitations,
as shown in Table~\ref{tbl:fn-systems}.

Specifically, existing systems fall short in their support for
cross-platform federated learning on Android and iOS devices.
FedML~\cite{he2020fedml},
Microsoft's Project Florida~\cite{madrigal2023project},
and the FedScale benchmark~\cite{lai2022fedscale} primarily support Android.
On the other hand,
Flower~\cite{beutel2020flower,mathur2021ondevice} and
Syft~\cite{ryffel2018generic,Ziller2021,hall2021syft}
provide software development kits for both Android and iOS.
However, Flower software development kits only provide communication support,
and do not provide support for on-device training;
it also provides incompatible communication formats in
the Android software development kit versus the iOS software development kit,
so that models from the two platforms cannot be aggregated together.
Syft only uses CPU for on-device training, and lacks hardware acceleration;
this is undesirable because on-device training is slow and energy-consuming
without hardware acceleration.
Therefore, building a federated learning system for smart campus applications is
still an open challenge.

\begin{table}\begin{center}\label{tbl:fn-systems}
    \begin{tabular}{lccccc}
Functionality         & FedML     & Project Florida   & Flower    & Syft      & \textbf{\fedkit} \\
\hline
Android-Only          & \ding{51} & \ding{51} & \ding{51} & \ding{51} & \ding{51}       \\
iOS-Only              & \ding{55} & \ding{55} & \ding{51} & \ding{51} & \ding{51}       \\
Cross-Platform Aggregation  & \ding{55} & \ding{55} & \ding{55} & \ding{51} & \ding{51}       \\
\hline
Training Acceleration & \ding{51} & \ding{51} & \ding{51} & \ding{55} & \ding{51}       \\
MLOps                 & \ding{51} & \ding{51} & \ding{55} & \ding{55} & \ding{51}       \\
Open-Source Backend   & \ding{55} & \ding{55} & \ding{51} & \ding{51} & \ding{51}       \\
    \end{tabular}
    \caption{Functionality Comparison Among Federated Learning Systems With
        Smartphone Support.
    }
\end{center}\end{table}

In addition to seamless cross-platform support,
customizable and continuous deployment in production is crucial.
While FedML and Project Florida support machine learning operations,
their proprietary backends limit full customization.
In contrast, open-source solutions like Flower and Syft enable
full customization
but are more suitable for single experiments rather than continuous deployment.

To support \fedcampus' use case, we developed our own custom solution \fedkit,
a software developed kit designed to enable \textbf{cross-platform}
federated learning research on Android and iOS devices.
\fedkit introduces two major contributions:

\begin{enumerate}
\item \fedkit provides \textbf{modularized libraries} to convert Python-based models,
    and train and aggregate them across platforms.
    These libraries can serve as a foundational resource for
    on-device FL or even ML systems.
\item The FL workflow in \fedkit
    enables flexible \textbf{machine learning operations} from
    the backend in production.
    This innovation encourages other open-source solutions to
    rival proprietary services in deployment support.
\end{enumerate}

\section{Federated Analytics}

\section{Healthcare Data Analytics}
