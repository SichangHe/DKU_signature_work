\section{Federated Learning And \fedcampus}

A significant subfield of artificial intelligence,
machine learning focuses on developing intelligent agents capable of adapting to
new situations based on historical data. In this adaptive process,
known as training, an intelligent agent analyzes training data,
enabling it to make predictions or inferences when confronted with unseen
situations.
This involves the construction and training of models—programs implementing
mathematical functions—by adjusting their parameters.

Federated learning
stands as a notable machine learning technique allowing distributed intelligent
agents to collaboratively train a model using their local
data~\cite{mcmahan2017communication,yang2019federated}.
The main motivation for this approach is to preserve user privacy and eliminate
the need to transfer raw data. Traditionally,
training machine learning models typically requires large amounts of centralized
data,
leading to mass collection of personal data and therefore privacy invasion.
% TODO: Cite.
In contrast,
federated learning is particularly suitable for scenarios where training data
cannot be shared centrally due to privacy restrictions.

This introduces the proposal of \fedcampus.
\fedcampus proposes a privacy-preserving data platform for smart university
campus built upon techniques including federated learning.
\fedcampus aims to analyze private data of its participants,
mainly university students, staff, and faculty,
without compromising their privacy.
This task is particularly suitable for federated learning,
which finds particular relevance in scenarios where the private data of
individual intelligent agents forms the crux of an machine learning problem.
For example, early practical applications of federated learning,
as proposed in~\cite{bonawitz2019towards}, include item selection and ranking,
personalized content recommendations,
and next-word prediction for smart keyboards on smartphones.
These applications leverage private data from user interactions with keyboards,
highlighting the essential role of federated learning in preserving user privacy
while enabling collaborative model training.

In contrast to previous federated learning applications,
\fedcampus is unique in its focus on real-world use cases and openness.
In the current phase, we conduct experiments on real participants' health data,
recorded by the smartwatches they wear. However,
being oriented to real-world use also poses significant challenges.
While several existing federated learning systems primarily cater to simulations
on desktop computers or cloud
servers~\cite[e.g.,][]{bonawitz2019towards,tff,caldas2018leaf,ma2019paddlepaddle,liu2021fate,openfl_citation},
they do not directly align with our real-world use case,
necessitating a unique approach tailored to our experiments.

\section{Federated Learning on Smartphones}

The extensive reach to private data on personal smartphones makes them ideal
participants in many federated learning applications.
Recent applications include vision-based product quality
inspection~\cite{bharti2022edge}
using an iOS application and short messaging service spam classification on Android
devices~\cite{sriraman2022device}.
These examples underscore the diverse range of practical applications that
leverage federated learning on mobile platforms.

Despite the theoretical advantages of federated learning on smartphones,
the practical deployment of its applications is still limited.
The concept of federated learning was introduced as early as
2016~\cite{mcmahan2017communication},
and its first effective and widely recognized application was announced in
2019~\cite{bonawitz2019towards}; however, even in 2023,
practical federated learning applications on smartphones remain scarce.
This scarcity can be attributed to multiple challenges that practical federated
learning encounters, including privacy and security concerns,
training efficiency and performance issues,
and heterogeneity problems~\cite{wen2023survey}.

Several systems have emerged to fulfill the demand for federated learning
on smartphones, but they still exhibit important limitations,
as shown in Table~\ref{tbl:fn-systems}.

Specifically, existing systems fall short in their support for
cross-platform federated learning on Android and iOS devices.
FedML~\cite{he2020fedml},
Microsoft's Project Florida~\cite{madrigal2023project},
and the FedScale benchmark~\cite{lai2022fedscale} primarily support Android.
On the other hand,
Flower~\cite{beutel2020flower,mathur2021ondevice} and
Syft~\cite{ryffel2018generic,Ziller2021,hall2021syft}
provide software development kits for both Android and iOS.
However, Flower software development kits only provide communication support,
and do not provide support for on-device training;
it also provides incompatible communication formats in
the Android software development kit (encoded in raw bytes)
% TODO: Cite
versus the iOS software development kit (encoded in the Pickle format),
so that models from the two platforms cannot be aggregated together.
Syft employs its custom on-device training implementation,
which only leverages the CPU, and lacks hardware acceleration;
this is undesirable because on-device training is slow and energy-consuming
without hardware acceleration.
Therefore, building a federated learning system for smart campus applications is
still an open challenge.

\begin{table}\begin{center}\label{tbl:fn-systems}
    \begin{tabular}{lccccc}
Functionality         & FedML     & Project Florida   & Flower    & Syft      & \textbf{\fedkit} \\
\hline
Android-Only          & \ding{51} & \ding{51} & \ding{51} & \ding{51} & \ding{51}       \\
iOS-Only              & \ding{55} & \ding{55} & \ding{51} & \ding{51} & \ding{51}       \\
Cross-Platform Aggregation  & \ding{55} & \ding{55} & \ding{55} & \ding{51} & \ding{51}       \\
\hline
Training Acceleration & \ding{51} & \ding{51} & \ding{51} & \ding{55} & \ding{51}       \\
MLOps                 & \ding{51} & \ding{51} & \ding{55} & \ding{55} & \ding{51}       \\
Open-Source Backend   & \ding{55} & \ding{55} & \ding{51} & \ding{51} & \ding{51}       \\
    \end{tabular}
    \caption{Functionality Comparison Among Federated Learning Systems With
        Smartphone Support.
    }
\end{center}\end{table}

In addition to seamless cross-platform support,
customizable and continuous deployment in production is crucial.
While FedML and Project Florida support machine learning operations,
their proprietary backend servers limit full customization.
In contrast, open-source solutions like Flower and Syft enable
full customization
but are more suitable for single experiments rather than continuous deployment.
For \fedcampus' purpose,
we require both the ability to conduct machine learning operations,
and to self-host and customize the backend server in production.

To support \fedcampus' use case, we developed our own custom solution \fedkit,
an open-source software developed kit\footnote{
    Available at \url{https://github.com/FedCampus/FedKit}.
} designed to enable \textbf{cross-platform}
federated learning research on Android and iOS devices.
\fedkit introduces two major contributions:

\begin{enumerate}
\item \fedkit provides \textbf{modularized libraries} to convert Python-based models,
    and train and aggregate them across platforms.
    These libraries can serve as a foundational resource for
    on-device federated learning or even ML systems.
\item The federated learning workflow in \fedkit
    enables flexible \textbf{machine learning operations} from
    the backend in production.
    This innovation encourages other open-source solutions to
    rival proprietary services in deployment support.
\end{enumerate}
