\section{Accessing the Data}

On Android, we use the Huawei Health Kit API Java package to request
Huawei Health Kit for the data.
This API allows access to all the collected raw health data points.
We call the API from using Kotlin,
a programming language compatible with Java,
and aggregate the data into a daily summary.
% TODO: What are the data types.
We used Kotlin to integrate with Huawei Health Kit API because
we found it to be more concise and maintainable than Java while
having complete interoperability.
A special pop-up authentication page is also implemented to request the user for
permissions to access their health data, as required by Huawei.

On iOS, we indirectly access Apple HealthKit through Flutter's
Health~\cite{flutterhealth} package.
The Health package provides a unified interface to access health data,
and notifies iOS to request permissions from the user to access the health data.
Since the Health package can also request health data from Google Fit on
Android, we provide an optional feature to use health data from Google Fit to
experimentally accommodate participants who use Android smartphones and
do not use Huawei smartwatches.

In our internal testing, we noticed that accessing the data could be slow,
especially when requesting data using the Huawei Health Kit API.
To mitigate this issue, we implemented a local cache database.
Before we request data from either of the health kits,
we first check the cache database for the data.
If the data is not in the cache database,
we request the data from the health kit and update the cache database.
% TODO: Not implemented yet.
We also implemented a background task to periodically update the cache database,
and the users can also force an update by refreshing the health page in
the application.

\section{The \fedcampus Application}

Most of the \fedcampus Application is implemented in Dart using
the Flutter application framework, with a few exceptions.
The data access on Android and the federated learning on-device training part
calls into platform-specific native code under Flutter's mechanism.

We leverage Flutter's strong support for
building attractive user interface to fulfill the need to
create a user-friendly application.
Additional, HTTP request libraries and JSON libraries in Dart are used to
communicate with the backend,
and an SQL library is used to manage the cache database.

\section{\fedkit}

Overall, \fedkit powers a federated learning system that
consists of a single backend server and a set of mobile application clients.
The clients train their local models on-device with their local data,
and exchange ML models and parameters with the backend server to conduct
federated learning.

\subsection{Cross-Platform Federated Learning Model Pipeline}

% TODO: Mention telemetry.

% TODO: Explain exactly how we record the model information.

% TODO: Explain exactly how we manipulate ProtoBuf.

\subsection{Machine Learning Operations}

% TODO: Mention Django
