\section{Participant Requirements}

Our involved data access setup results in involved participant requirements.

In practice, accessing the data via the Huawei Health Kit on Android only works
well for participants using Huawei smartphones.
On Huawei smartphones,
the operating system automatically periodically synchronizes health data from
the connected Huawei smartwatch to the Huawei Health Kit.
Therefore, when these participants launch our FedCampus Application,
the app can immediately access the latest data from the Huawei Health Kit.

However, on non-Huawei Android smartphones,
the synchronization from the smartwatches only happens when
the Huawei Health app is on the foreground,
unless participants install Huawei Management System Core (HMS Core),
a background service app.
Alpha testers reported that the HMS Core is battery-consuming,
and is sometimes shut down by the operating system on smartphones made by
Xiaomi and other companies.
Additionally, Huawei apps are not available on the Google Play Store,
so some participants also have to install the Huawei AppGallery Application.
This makes an undesirable experience where the participants are required to
install four different apps.
Despite the inconveniences, once the participants have installed the apps,
they do not need to conduct this process again.

On iOS, participants can download the Huawei Health app from the App Store,
but it does not synchronize data from the Huawei smartwatch in the background.
Participants have to manually open the Huawei Health app,
and leave it in the foreground to synchronize the data both from
the smartwatches to the Huawei Health Kit,
and from the Huawei Health Kit to the Apple HealthKit.
It takes a significant amount of time for Huawei Health Kit to
synchronize the data to Apple HealthKit,
especially for the sleep time data;
as a result, it is common for the FedCampus Application to miss sleep data as
it could not read it from Apple HealthKit.

Due to these practical limitations,
we first prioritized Huawei smartphone users, then iPhone users,
for our initial launch.
